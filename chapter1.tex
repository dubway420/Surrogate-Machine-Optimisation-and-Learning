\chapter{Introduction}
\label{cha:intro}

\section{Outline}


The overarching motivation behind this thesis is to develop machine learning (ML) techniques for the benefit of safety analysis in the UK nuclear energy sector. The objective of this research was to develop machine learning models which can perform the same functionality as existing industry standard models but with reduced computational intensity.  
\\ 

\noindent
The UK nuclear industry is dominated by the advanced-gas cooled reactor (AGR), a design which uses graphite as a structural component and carbon dioxide gas as a coolant. The AGR differs in design to most reactors elsewhere in the world, meaning there is a lack of international research to draw on concerning operation and safety. Instead, all analysis must be preformed domestically, meaning that research in this field is intensive. The situation is further complicated by the complexity of AGR safety analysis, involving thousands of different components in various scenarios and configurations. The combination of these factors means that those responsible for demonstrating the safety of AGR reactors face challenges in terms of computational requirements whilst having a time sensitive task. 
\\ 

\noindent
The research question to be answered in this thesis concerns whether the difficulties faced within the UK nuclear industry can be alleviated through the use of machine learning techniques. To this end, a key research objective is the production of surrogate machine learning models (SMLMs). SMLMs are produced with the intention of retaining the functionality of an existing model i.e. it produces similar outputs from the same inputs. The intended advantage of SMLMs over the models they surrogate is low computational cost, hence fast inference of outputs from inputs. Compared to the original models, which may take many hours or even days to complete their analysis, SMLMs once trained can produce outputs in seconds using the same computational hardware. 
\\ 

\noindent
Through the development of SMLMs for this research work, two articles were written and published in peer-reviewed academic journals. The first article, presented in the journal of Nuclear Engineering and Design, details the generation and preparation of data, as well as the process of developing an SMLM for our purpose \cite{jones2022surrogate}. This article highlights how some methods were found to be highly effective, such as convolutional neural networks and regularisation. It was noted that through a process of feature selection, not only can performance of the model be improved, but insights into the underlying nature of the data space can be observed. The second article, presented in IEEE Access, details improvements to the method outlined in the first article. Three methods were exploited: data augmentation, custom loss functions and transfer learning. Each of these methods have seen previous exploitation within the field of machine learning, however, we implement them here in a novel way. 


\section{Research Objectives}

The key objectives of this research project are as follows.

\begin{enumerate}
	\item \textbf{Development and Optmisation of SMLMs} The overarching objective is to produce a machine learning model which surrogates the function of an existing standard engineering model. The aim is to reduce computational intensity whilst retaining a high level of accuracy/functionality. 
	
	\item \textbf{Data Visualisation Tools} With the data space for this research area being highly complex and multi-dimensional, a secondary objective is to explore the data through visualisation techniques. To this end, bespoke computational tools will need to be produced. The benefit of data visualisation for this research task is two fold: 1) It will help inform the process of machine learning model development and optimisation 2) Insights into the data space will be informative to a wider engineering audience.
	
	\item \textbf{Data Insights Through ML} Through the process of development and optimisation of SMLMs, insights into the underlying nature of the data space itself. These insights will benefit the development of SMLMs but also have value in and of itself. For example, relationships between parts of the data may yield insights which are useful from a wider engineering or safety perspective. 
	
	
\end{enumerate}



\section{Thesis Structure}

The content of this thesis is presented as follows:

\begin{enumerate} 
	
	\item \textbf{Nuclear Engineering Background} Technical information concerning the AGR, traditional engineering calculations, existing modelling techniques. 
	
	\item \textbf{Machine Learning Background} Details of machine learning techniques and how they work. Differences between various methods.
	
	\item \textbf{Dataset Framework, Exploration and Analysis} Details of the programmatic dataset framework, as well as an exploration of the dataset using visualisation techniques and statistical methods.
	
	\item \textbf{Preliminary Machine Learning Experiments} Initial experiments performed with the intention of exploring the problem space and testing ideas. 
	
	\item \textbf{Development and Analysis of a Surrogate Machine Learning Model} This chapter is based on a published research paper. It covers the development and optimisation of a surrogate machine learning model. 
	
	\item \textbf{Methods to Improve Surrogate Machine Learning Model Performance} This chapter is based on another published work titled "Data-driven Approaches to Surrogate Machine Learning Model Development". It covers the adaption of existing machine learning methods to the research topic at hand.
		
\end{enumerate}
	
	


